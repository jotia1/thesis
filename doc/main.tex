\documentclass[12pt,openany,a4paper]{book}
\usepackage{graphicx}	% if you want encapsulated PS figures.
\usepackage{wrapfig}
\usepackage{verbatim}
\usepackage{hyperref}
\usepackage{siunitx}
\usepackage{amsmath}
\usepackage[skip=2pt]{caption} % Bring captions closer to figure

% Rename bibliography to reference
\usepackage[nottoc,notlof,notlot]{tocbibind} 
\renewcommand\bibname{References}

\usepackage{cite}

\graphicspath{{imgs/}}

% Number subsections but not subsubsections:
\setcounter{secnumdepth}{2}
% Show subsections but not subsubsections in table of contents:
\setcounter{tocdepth}{2}

\pagestyle{headings}		% Chapter on left page, Section on right.
\raggedbottom

\setlength{\topmargin}		{-5mm}  %  25-5 = 20mm
\setlength{\oddsidemargin}	{10mm}  % rhs page inner margin = 25+10mm
\setlength{\evensidemargin}	{0mm}   % lhs page outer margin = 25mm
\setlength{\textwidth}		{150mm} % 35 + 150 + 25 = 210mm
\setlength{\textheight}		{240mm} % 

\renewcommand{\baselinestretch}{1.2}	% Looks like 1.5 spacing.

% Stop figure/tables smaller than 3/4 page from appearing alone on a page:
\renewcommand{\textfraction}{0.25}
\renewcommand{\topfraction}{0.75}
\renewcommand{\bottomfraction}{0.75}
\renewcommand{\floatpagefraction}{0.75}

% AIDS TO CROSS-REFERENCING (All take a label as argument):
\newcommand{\eref}[1] {(\ref{#1})}		% (...)
\newcommand{\eq}[1]   {Eq.\,(\ref{#1})}		% Eq.~(...)
\newcommand{\eqs}[2]  {Eqs.~(\ref{#1}) and~(\ref{#2})}
\newcommand{\dfn}[1]  {Definition~\ref{#1}}	% Definition~...
\newcommand{\thrm}[1] {Theorem~\ref{#1}}	% Theorem~...
\newcommand{\lem}[1]  {Lemma~\ref{#1}}		% Lemma~...
\newcommand{\fig}[1]  {Fig.\,\ref{#1}}		% Fig.~...
\newcommand{\tab}[1]  {Table~\ref{#1}}		% Table~...
\newcommand{\chap}[1] {Chapter~\ref{#1}}	% Chapter~...
\newcommand{\secn}[1] {Section~\ref{#1}}	% Section~...
\newcommand{\ssec}[1] {Subsection~\ref{#1}}	% Subsection~...

% AIDS TO FORMATTING:
\newcommand{\teq}[1]	{\mbox{$#1$}}	% in-Text EQuation (unbreakable)
\newcommand{\qed}	{\hspace*{\fill}$\bullet$}	% end of proof

% MATHEMATICAL TEMPLATES:
% Text or math mode:

% ORDINAL NUMBERS:
\newcommand{\ith}	{\ensuremath{i^{\rm th}}}
\newcommand{\jth}	{\ensuremath{j^{\rm th}}}
\newcommand{\kth}	{\ensuremath{k^{\rm th}}}
\newcommand{\lth}	{\ensuremath{l^{\rm th}}}
\newcommand{\mth}	{\ensuremath{m^{\rm th}}}
\newcommand{\nth}	{\ensuremath{n^{\rm th}}}

% NB: These have not been tested since being modified for LaTeX2e.
\newcommand{\pack}	{\hspace{-0.08em}}
\newcommand{\Pack}	{\hspace{-0.12em}}
\newcommand{\Hz}	{\ensuremath{\rm\,H\pack z}}
\newcommand{\cm}	{\ensuremath{\rm\,cm}}
\newcommand{\s}		{\ensuremath{\rm\,s}}
\newcommand{\ms}	{\ensuremath{\rm\,m\pack s}}
\newcommand{\us}	{\ensuremath{\rm\,\mu s}}

\begin{document}

\frontmatter
% By default, frontmatter has Roman page-numbering (i,ii,...).

\begin{titlepage}
\renewcommand{\baselinestretch}{1.0}
\begin{center}
\vspace*{35mm}
\Huge\bf
		Analysing time-stepped \\
        nerual networks with temporally \\
        rich frame-based data \\
\vspace{20mm}
\large\sl
		by\\
		Joshua Arnold
		\medskip\\
\rm
		School of Information Technology and Electrical Engineering,\\
		The University of Queensland.\\
\vspace{30mm}
		Submitted for the degree of\\
		Bachelor of Engineering
		\smallskip\\
\normalsize
		in the field of Software Engineering
		\medskip\\
\large
		June 2016.		
\end{center}
\end{titlepage}

\cleardoublepage

\begin{flushright}
	ADDRESS LINE 1\\
	ADDRESS LINE 2\\
	Tel.\ (07) nnnn nnnn\\
	\medskip
	\today
\end{flushright}
\begin{flushleft}
  Prof Paul Strooper\\
  Head of School\\
  School of Information Technology and Electrical Engineering\\
  The University of Queensland\\
  St Lucia, Q 4072\\
  \bigskip\bigskip
  Dear Professor Strooper,
\end{flushleft}

In accordance with the requirements of the degree of Bachelor of
Engineering in the division of 
Electrical Engineering,
Electrical and Biomedical Engineering,
Electrical and Computer Engineering,
Software Engineering,
Mechatronic Engineering,
I present the
following thesis entitled ``Analysing time-stepped neural networks with temporally rich frame-based data''.  This work was performed under the supervision of Prof. Janet Wiles.

I declare that the work submitted in this thesis is my own, except as
acknowledged in the text and footnotes, and has not been previously
submitted for a degree at The University of Queensland or any other
institution.

\begin{flushright}
	Yours sincerely,\\
	\medskip
	\emph{}\\
	\medskip
	Joshua Arnold.
\end{flushright}

\cleardoublepage

% Dedication (if you want it):
\vspace*{70mm}
\begin{center}
\renewcommand{\baselinestretch}{1.0}
\sl
	To \ldots
\end{center}

\chapter{Acknowledgments}

Acknowledge your supervisor, preferably with a few short and specific
statements about his/her contribution to the content and direction of
the project.  If you collaborated with another student, acknowledge
your partner's contribution, including any parts of the thesis of
which s/he was the principal author or co-author; this information can
be duplicated in footnotes to the chapters or sections to which your
partner has contributed.  Briefly describe any assistance that you
received from technical or administrative staff.  Support of family
and friends may also be acknowledged, but avoid sentimentality---or
hide it in the dedication.

\cleardoublepage

%%%%%%%%%%%%%%%%%%%%%%%%%%%%%%%%    ABSTRACT    %%%%%%%%%%%%%%%%%%%%%%%
\chapter{Abstract}

\begin{itemize}
\item Need to nail down thesis of the thesis... \hfill

\item Linear patterns can be extracted from event-based data represented as a two dimensional temporally decayed images in time-stepped neural network models. \hfill

\end{itemize}

Event-based sensing offers many advantages in the field of artificial vision however few algorithms are able to leverage the data fully. 
This is due to a disparity between traditional frame-based algorithms developed for 2D images and the 3D nature of event-based data. 
A new event-based dataset of linear movement is presented as a benchmark for simple models.
This dataset is then used to evaluate the representational power of 2D temporally blurred images in frame-based neural networks. 
By retaining a time based blur, frame-based networks are able to implicitly encode for temporal information in predictions. 

\tableofcontents

\listoffigures

\listoftables

\cleardoublepage

\mainmatter
% By default, mainmatter has Arabic page-numbering (1,2,...).

\chapter{Introduction}

%The introductory chapter describes the importance of the field and the
%scope and significance of your project.  It usually ends with an
%overview of the remainder of the thesis.

%Notice that Arabic page numbering begins with Chapter 1.  Preceding
%pages (known as ``frontmatter'') have Roman numbering.  The
%\texttt{book} document class in \LaTeX\ follows this numbering
%convention by default (see Lamport~\cite{lamport}, p.\,80).





%% Broard intro to topic, where did this come from, what is the problem, 
Vision is the primary sense used by many natural agents to gather information about their environment.
It would seem to follow that vision should be an important percept for artificial agents.
%It would seem to follow that artificial agents should also be able to leverage visual sensors in a similar way.
% TODO Is digital systems the right term to use here??
However the apparent ease with which natural systems process visual information does not translate to standard digital arcitectures, making vision a seldom used sense. 
In real time applications vision is often impractical out due to the computational load required to extract meaningful information. 
%In its place alternatives such as Infra-red for distance sensing or L.I.D.A.R. for mapping are substituted.
% TODO gonna need a refernce below
In a more theoretical environment there are still many challenges in extracting meaningful information from standard vision data due to variations in lighting, orientation, position, scale etc.


%% Breif into to neural nets
%   Why they are better than hard coded rules -> they can learn
A simplistic approach to vision processing would entail defining exact features of the object to be recognised.
This would quickly become overwhelming though as the system decends into considering a seemlingly endless list of special cases and permuations that a single object could take in an image.
% TODO will need a reference to standad analysis algoithms
%Numerous algorithms and heuristics have emerged making meaningful analysis possible in some circumstances such as ****** Color filtering / normal face detection?? / Canny edges? *******.
%Unfortunately these struggle to generalise to arbitrary object classification or prediction.
% TODO reference some state-of-the-art networks
%A promising alternative, Neural Networks, currently have state-of-the-art performance on many of the public benchmark datasets.
Alternatively an algorithm designed to detect a specific object using heuristics or traditional vision processing methods could be used.
Unfortuneately these only solve a smaller part of the processing problem, such as canny edge detection, or struggle to generalise to object other than that they were designed for, such as ** face detection **.
Neural networks offer a general solution to the problem of classification and prediction.
Rather than relying on expertly designed heuristics or algorithms a neural network is randomly initialised and after repeated presentations of some stimuli can adjust it's parameters to minimise a loss function. 
This ability to self adjust and decide which parts of the stimuli are important has lead to systems capable of learning representations far more complex than an expert would have been able to design explicitly. 


%% Breif into to Event-based sensors
%   fast, low power, sparse, biologically realistic 

%% Why the DVS with NN's makes sense


%\section{}
             % 3
\chapter{Literature - Sensors, Processing, Networks}

%%%%%%%%%%%%%%%%%%%%%%%%%%    ASYNC SENSORS     %%%%%%%%%%%%%%%%%%%%%%%%%%%%%%%%%

\section{Asynchronous Vision Sensors}  % 1 page
%Lots of different types of dynamic vision sensors \cite{delbruck2010activity}
%Why would we want a DVS? microparticle tracking \cite{ni2012asynchronous}
%Neuromorphic systems are becoming more wide spread \cite{delbruck2014research}
% Introduction to AVS 
%Introduce different kind of senssors which asynchronously fr
Asynchronous Vision Sesnors (AVS) offer a frame--free, event--driven approach to capturing vision information\cite{delbruck2010activity}. 
These sensors continuously output light intensity changes from a scene in the form of precisely timed (sub \ms) events. 
Each pixel in the sensor array acts independently based on the light in its receptive field resulting in a sparse, stimulus--based output. 
A recent spike in technology facilitated the development of many alternative asynchronous vision sensors with varying properties and speeds\cite{delbruck2014research}.
These systems are united in an attempt at emulating the efficient, high--speed, event--based spiking behaviour in biological vision systems\cite{delbruck2010activity}.
%Environments requiring low-power, low-latacy and dynamic range are well suited 
Asynchronous vision sensors are well suited to environments requiring dynamic range, low-power and low--latency sensing, such as microparticle tracking\cite{ni2012asynchronous}, robotics\cite{roboGoalie2013} or motion tracking and classification\cite{Lee2014, reverter2015neuromorphic}.

% How do they work
In contrast to traditional vision systems which densely sample the world at discrete time intervals, AVSs follow a stimulus driven paradigm only recording changes in the environment, meaning redundant information in the scene (e.g. stationary items or backgrounds) are not captured.  
The resulting data format, Address-Event Representation (AER)\cite{mahowald1992vlsi}, is in the form $\textless (x, y), t, p \textgreater$, where $(x, y)$ is the pixel location of a change, $t$ is the precise time of the change and $p$ is the polarity indicating if the light intensity change was brighter (positive event) or dimmer (negative event)i\cite{delbruck2010activity}. 
% TODO really should find a AER specific paper for this
This dramatically different data representation means new approaches to vision processing will be required to extract meaningful information from AER sensors\cite{akolkar2015can}. 

% They why. What are the advantages and disadvantages
Asynchronous vision sensors offer a new paradigm in which to approach capturing vision information, bringing with it many new opportunities and challenges. 
The low--redundancy asynchronous nature of these sensors allows low--power, low--computation processing in a more biologically realistic setting.



\subsection{Example sensors}
%Tobi made a camera \cite{delbruck2008}
%Discuss the Dynamic vision sensors, what they are capable of
%Used in stuff like \cite{delbruck2007fast}
%They also have a new camera called the DAVIS \cite{DAVIS}
The Dynamic Vision Sesnor (DVS) is an AVS capable of registering events to temporal precision of 15\us\cite{delbruck2008}. 
It has a 128x128 pixel array with \textgreater 120\textit{dB} dynamic range.
The Dynamic and Active Pixel Vision Sensor (DAVIS) is the newer model of the DVS with a 240x180 pixel array, 3\us temporal precision and 130\textit{dB} dynamic range\cite{DAVIS}. 
Additionally the DAVIS has circuitry (the active pixel sensor) enabling it to capture static scene illumination values like a standard camera.
Further an inbuilt inertial measurement unit (IMU) means movement information can be simultaneously recorded and used in processing. 
An alternative sensor is the asynchronous time-based image sensor (ATIS) with a temporal resolution of 10\us (at \textgreater100Lx)\cite{posch2011qvga}.
Like the DAVIS the ATIS is also capable of capturing scene illumination as well as events.  





%%%%%%%%%%%%%%%%%%%%%%%%    STANDARD VISION     %%%%%%%%%%%%%%%%%%%%%%%%%%%%%%%%%
\section{Frame-based vision processing}   % 1 page
%Standard videos and frame based approaches

% Introduction to standard vision processing 
Standard vision processing has long been based on processing full 2D frames using well understood techniques such as Canny Edge detection\cite{canny1986computational}. 
Many techniques exist for various tasks such as face detection\cite{viola2004robust}, object tracking using kernels\cite{comaniciu2003kernel} or object classification\cite{krizhevsky2012imagenet}.
The method of these approaches varies significantly from gradient based computation in Canny Edge detection to kernel based object tracking to deep neural networks.
Yet they all make the same implicit assumption that vision information is temporally discretised into the frame-rate of the vision sensor used in recording. 

Frame-based techniques and associated applications are thus bound by the limitations of the recording device with respect to data speed and quality. 
In particular if a standard 30 fps camera is used a realtime system must wait 33ms between frames (excluding any processing time required). 
Using a higher frame--rate recording device has a bound on performance as this necessarily means there will be more data (much of which is redundant) to process, such that the processing time of the system becomes the bottle neck. 
Further image quality in frame-based systems is proportional to the amount of processing required to extract features.
Despite limitations from recording devices, many impressive results have come from these techniques, in particular deep neural networks\cite{krizhevsky2012imagenet}.
However the computation required for these state-of-the-art systems makes their use on low--power, low--computation devices impractical. 



%%%%%%%%%%%%%%%%%%%%%%%%    EVENT-BASED PROCESSING     %%%%%%%%%%%%%%%%%%%%%%%%%%%%%%%%%
\pagebreak
\section{Event-based processing}     % 2 page
% Intro to what event based processing is (using events)
Event-based data brings with it the need for radically different processing paradigms compared to frame-based approaches\cite{perez2013mapping, martin2015spiking, tan2015benchmarking}.
Conventional sensors generate large amounts of redundant data which is computationally expensive to process, making efficient event-based data an attractive option\cite{vanarse2016review}. 
Event-based processing is not limited to vision sensors however, as discussed in \cite{vanarse2016review} neuromorphic auditory and olfactory sensors also exist using the AER format. 
Challenges facing event-based vision sensors are shared among these other neuromorphic sensors as well as more common (and low cost) sensors (like IMUs for gait cycle measurments\cite{fida2015pre}).

Many standard machine learning techniques have implicit assumptions that data is discretised into uniform time samples or that temporal information is not present/important.
There have been attempts to integrate temporally rich data with standard frame-based approaches such as the recurrent Temporal Restricted Boltzmann Machine (TRBM)\cite{sutskever2009recurrent} to model motion capture data or using deep-belief networks with spiking systems\cite{Neftci2014, pedroni2013neuromorphic, OConnor2013}.
More suited to dealing with event-based data are the family of Spiking Neural Networks (SNN)\cite{henderson2015spike, perez2013mapping}.
SNNs are harder to train than the frame-based counterparts, leading many to train frame-based networks and convert the trained model into an equivilant spiking network at some small performance drop\cite{perez2013mapping, pedroni2013neuromorphic, OConnor2013}.

Event-based processing is not limited to neural networks and other techniques are being adapted or created to take advantage of these new highspeed sensors\cite{ni2015visual}.
A \textit{RoboGoalie} was created as an example application leveraging the low-latency sensor to stop fast moving ping-pong balls entering a goal\cite{delbruck2007fast}. 
A simple event clustering algorithm was sufficient to allow the RobotGoalie to track incoming balls and respond within 3\ms with a peak performance load of 4\% on a standard computer. 
This was conducted in a heavily constrained system though, where luminance was controlled, the DVS stationary and with a constant background. 
Strict weight, power and computational requirements of a quadrotor form a near-perfect environment in which traditional cameras fail and event-based sensors shine\cite{mueggler2014event}.
These low-power and low-computation features of event-based sensors make them particularly attractive for applications in mobile robotics and navigation\cite{weikersdorfer2013simultaneous, milford2015place}.
These constraints have led to focused efforts on developing efficent algorithms to calculate odometry \cite{censi2014low}, visual flow \cite{benosman2014event} and corner detection\cite{clady2015asynchronous}.

% More on applications here then next para will support why they make sense

The systems described have shown that through careful datastructures and representations, learning models are able to leverage temporal information from the precise spike timings of event-based data. 
It is well known the eye does not act as a traditional vision sensor capturing and transfering full frames to the brain but rather sends only relevant visual information in the form of spikes\cite{delbruck2010activity}.
In an attempt to mimic this spiking behaviour some have attempted to convert frame-based benchmark datasets into spiking equivilants using grayscale values to produce rate-coded spiking outputs, while others have suggested why this may be problematic \cite{akolkar2015can} and suggest guidelines to ensure dataset quality\cite{tan2015benchmarking}.
Studies have found that coding schemes using the precise timing of events may be more biologically realistic and account for situations where processing happens too quickly to be explained by rate-coded information transfer\cite{thorpe2001spike}. 
It was similarly found that the precise spike timings of events in event-based data has a significant contribution to the amount of information contained in the recording\cite{akolkar2015can}.
% NOTE: Thus precise spike timings are important so maybe using a screen is bad. 


%What other processing has already been done with them (things like Motion cones). 
%Example usage of DVS is fast motor control by \cite{delbruck2007fast}
%Discuss frame accumulation by other groups
%Example of event-based visial flow calculation by fitting local plans on incoming events\cite{benosman2014event} (what else is this lab doing?)
%Calculating odometry using event based information (application) \cite{censi2014low}
%event-based corner detection\cite{clady2015asynchronous}


\subsection{Visual information representations}

% The eye can process well
The ease with which biological systems can process visual information suggests it should be a key percept in artificial agents, but it is not.
% But uses funny data formats
The challenges limiting artificial vision systems may be in part due to the fundamental differences between the biological equivilant in both recording and data format.
% DVS is like the eye
Neuromorphic vision sensors have started to emerge closing the gap between artificial systems and biology\cite{mahowald1992vlsi}.
% and has similar data formats
These sensors record and output data which much more closely resembles the retina\cite{akolkar2015can}. 
% Could convert to Frames
Although more biologically realistic, the data formats from these sensors are fundamentally different to frame-based representations prohibiting standard processing techniques. 
Converting between events and frames can be as simple as collecting all events from a time period into a frame as in \cite{kogler2009bio, schraml2010dynamic} or involve more complex calcuations about the events usefulness\cite{mueggler2015lifetime}.
Inversely, there has been work converting from frames back to realistic spike-train\cite{afshar2013ripple} showing this conversion is not a simple mapping but contains losses in conversion\cite{akolkar2015can}. 
%   Not-bio but it works
Accumulating events into frames works with current processing paradigms but departs from biology which is likely using precise neuron firings as representations\cite{akolkar2015can}.
% If we want to process well we must learn to process data formats too
To achieve biological system level performance is may be necessary to preserve additional qualities of the biological data format and processing. 
% We have AER
Perhaps the most biologically realistic format considered for processing is the AER format \cite{mahowald1992vlsi}.
%   But this is limited in number algorithms
This relatively new format has few algorithms that can fully leverage the structure, some worthy mentions include a high-speed pencil balancing robot \cite{conradt2009pencil}, a robotic goalie \cite{roboGoalie2013} and scene reconstruction with super-pixel resolution\cite{kim2008simultaneous}.
%   Could use Spiking neural networks (but hard to train)
% Temporal surface
A middle ground between direct frame accumulation and AER formats is the use of a short term window in which events have some influence.
Memory surfaces are present in the literature in forms such as gaussians approximating position \cite{conradt2009pencil}, queues of events \cite{ni2012asynchronous} or as this work will investigate, functions of temporal difference \cite{afshar2016investigation}.


% 

%Disucss how this is more like the eye and the advantages of this model of processing \cite{mahowald1992vlsi}

%Interest in event-based processing and sensors stems from the analogies to biological vision and processing\cite{mahowald1992vlsi}. 
%Necessarily vision sensors more like biological retinas must use  essence of neural processing data structures which closely resemble real .

%Rumelhart's back prop might not actually be biologically realistic but effective in learning regardless \cite{Rumelhart1986}.
%Discusses the biological realism of using spikes \cite{akolkar2015can}
%What does this paper say about representations maybe bio isn't always an answer \cite{fida2015pre}

%Temporal surfaces \cite{afshar2016investigation}



%%%%%%%%%%%%%%%%%%%%%%%%%%%%%%  NEURAL NETWORKS   %%%%%%%%%%%%%%%%%%%%%%%%%%%%%%%

\section{Neural Networks}     % 1 page
% Neural networks are loosely based on the brain
Artificial neural networks have origins as software implementations of brain functions\cite{mcculloch1943logical}.
Several seminal breakthroughs, noteable being backpropagation (popularised by \cite{Rumelhart1986}), convolutional neural networks \cite{lecun1998gradient} and more recently deep learning \cite{schmidhuber2015deep} have made neural networks a powerful tool in many learning tasks. 
% Impressive performance (state of art) in image classification
In particular neural networks have become the state-of-the-art in many image recognition and classification tasks \cite{krizhevsky2012imagenet, szegedy2015going}.

% Composed of layers of processing units (neurons)
Neural networks are made up of many processing units each which does a small computation and produces output. 
% Type and function of units affects network dynamics
The function each individual processing unit (sometimes also called neurons) computes can have a significant affect on the network performance. 
%   perceptron
The first, and perhaps simplest, processing unit is the perceptron which computes a binary classification based on a weighted sum of inputs inputs. 
%   Sigmoid
%   ReLU
%   More complex LSTM units exist but not used

% Specifying weights impossible so we use back prop and SGD to refine weights
Adjusting internal parameters of the network cause it to compute different functions on the input, its performance can then be evaluated according to some loss function. 
% Loss functions play a big role
% Correct weight initialisation also



% Mention:
% Perceptions, sigmoidals, ReLUs, SGD, backprop
%Learning representations by back--propagating errors\cite{Rumelhart1986}

\subsubsection{Autoencoders}
% What they do and how they work to clean data/form representations


\subsubsection{Convolutional networks}
% Use in image processing 

\subsection{Deeper networks}   % 1 page
ResNet etc. % TODO find resnet reference
ImageNet etc.
This was some deep network work \cite{pedroni2013neuromorphic}
Oconor used deep networks \cite{OConnor2013}

\subsection{Spiking networks}    % 1 page
O'Connors work on spiking deep belief networks \cite{OConnor2013}
Using echo state networks to extract spatiotemporal features from event-based \cite{lagorce2015spatiotemporal}
Move to spiking networks accompanied by converstions from 2D to 1D temporal trains \cite{afshar2013ripple}
Discussion on 2D images and converting to spikes vs just using spiking sensor and the information preserved \cite{akolkar2015can}
Paper on spiking networks for vision tasks, usese the DVS and compares to convNets \cite{martin2015spiking}
Feeding a DVS output directly into spiking NN \cite{Bichler}
%Interesting learning rules for SNNs \cite{Bichler} not exaimined here though.

\subsection{Software frameworks}   % 1 page
Many software frameworks exist for implementing and running neural networks including; Caffe\cite{jia2014caffe}, Theano\cite{bastien2012theano}, Torch7\cite{collobert2011torch7} and Tensorflow\cite{abaditensorflow} plus many more.
Each package 

\section{Neural Networks with event-based data}
What is the standard approach, what have people tried and what is lacking. \cite{OConnor2013}  ** TOBI CAR WITH CONVOLUTIONS ** ** UWS temporal surfaces paper if published now ** ** AMYs work ** (who does amy reference?)
Using event-based data and feeding it into a network (Saeed) \cite{afshar2016investigation}

\subsection{Benchmark datasets}
Some benchmark datasets have been given \cite{Gibson2014} and **OTTHER** but represent complex strucutre in the real world.
Creates own plane dataset \cite{afshar2016investigation}
Creates own letter (on rolling board) and DMD projector dataset \cite{akolkar2015can}
Newest dataset of motion and visual navigation, emphasises the need for dataset \cite{barranco2016dataset}

%%%%%%%%%%%%%%%%%%%%%%%%%%%%%%%%%    SUMMARY    %%%%%%%%%%%%%%%%%%%%%%%%%%%%%%%%%%%%%

\section{Literature summary}      % 1 page
Revisit each section in a sentance or two and link them all together.

Evolving probabilistic spiking neural networks for spatio-temporal pattern recognition: A preliminary study on moving object recognition \cite{kasabov2011evolving}

Neuromorphic adaptations of restricted boltzmann machines and deep belief networks, These guys (in particular Pedroni) had something stuff \cite{pedroni2013neuromorphic}


Double check the use of this paper \cite{gil2014active}

How does this paper fit? applications? what are they doing? temporal surfaces maybe? \cite{davide2014high}

again high speed object tracking (diff authors) \cite{saner2014high} 

More somewhat unrelated, must just use about the methods of processing \cite{mueggler2015continuous}
         % 12
\chapter{Datacollection and pre-processing}

\section{The need for a dataset}

\section{Requirements}
Simple, ground truth, 

\section{Considerations}
Flashes on screen -> noise
Segmenting datasets
Line angles (Close to edge, just a tiny segment in corner)
Ground truth to actual recording consideration (mitigated through setup?)
Light invariance
Black on white and white on black, grey
Hot pixels

\section{Methodology}
What did i actually end up doing

\section{Final dataset specifications}


\section{Comments}
Looking back at the data what do i think?
Flat timestamps?? -> experimental hardware
Segmenting looses a lot on each side
No metric for validating ground truth against actual dot accuracy
Camera angles (parallel lines not so).




    % 5
\chapter{Processing - Accumulating the past and future}
\label{ch:preprocess}

\section{Aims}
Representations have a significant effect of what and how quickly a network can learn\cite{akolkar2015can}. 
Thus for any system to predict motion it must necessarily encode some representation of time. 
Traditional approaches to this have been to design Recurrent Neural Networks (RNNs) capable of storing some memory of previous states to inform future decisions.  
Alternatively, a system may make use of precise spike timings to encode information as in Spiking Neural Networks (SNNs). 
Many state-of-the-art learning techniques poorly encode time by making the implicit assumption that time is discretised into the frame-rate of the camera used to make a recording.
The decision then seems to be, work with mature state-of-the-art techniquesi, such as deep learning which are poorly able to capture temporal information, or RNN/SNNs better suited to process event-based data but with sub--optimal performance.   
However this need not be the case; to leverage the existing literature and use these state-of-the-art models with event-based data, a representation that these networks can accept must be used. 
This section will explore a method of representing event-based data in a frame-based structure while preserving as much temporal information as possible. 


\section{Method}
The simplest solution is to accumulate all the events in a given time slice into a 2D frame to recover images from the event-based data.
This accumulation is somewhat counter intuitive, as this forfeits all the temporal information which separates the DVS from traditional cameras and results in a blurry low resolution image. 

\begin{wrapfigure}{r}{0.6\textwidth}
    \centering
    \includegraphics[width=0.6\textwidth]{stimulus_and_accumulated.png}
    \caption{Example stimulus and a possible representation of the faded history}
    \label{fig:fadedhistory}
\end{wrapfigure}

An approach to this problem is to choose a distinguished point in time, then construct a 2D image in which each pixel's value is a function of its temporal distance to that distinguished point.
If only events that occurred prior to the distinguished point in time are considered the resulting image would represent a faded history of what has just happened. 
A faded history can be seen in figure \ref{fig:fadedhistory} in which a recent history of the scene (the moving dot from left to right) can be observed. 
Such images can be considered as a faded past, but are also known by many other equivalent names such as; temporal surface, decayed surface, time fields or time surface and these terms can be used interchangeably. 
There is no restriction as to why accumulation can only occur into the past; a similarly interesting image can be produced by accumulating into the future from a distinguished point in time.
These accumulated futures show where objects are moving. 
Consequently if a distinguished point is chosen and an image is accumulated into both the past and future from this point (as per figure \ref{fig:fadedhistory}) these two images can be used as input and labels in a neural netowrk. 


The data is now in a form which can be used in frame-based neural networks and some of the temporal information has been maintained.
Many questions still remain to be explored such as what functions should be used to decay images and with what parameters.


\section{Function specifics and implementation}
%Include a quick maths wise description of this function\\
%Include some resulting images and discussion about each image as to why it is good or bad \\

Two accumulation functions are considered in this work, the first and simplest being linear. 
It can be characterised by equation \ref{eq:linearDecay}, in which the gradient (k) directly affects how much of the past (or future) is considered. 
Here $\Delta t$ is the temporal difference between the closest event for a given pixel and the distinguished point in time. 

% TODO why is this 1/k??

\begin{equation}
 \label{eq:linearDecay}
    f(\Delta t) = 
    \begin{cases}
    -\frac{1}{k}  \Delta t + 1 & 0\leq \Delta t \leq k \\
    0 & Otherwise
   \end{cases}
\end{equation}

An exponential function was also considered as a way to focus on only the most recent history.
Characterised by equation \ref{eq:expDecay} the length of history considered is dependent on the parameter k used. 

\begin{equation}
 \label{eq:expDecay}
    f(\Delta t) = exp\left(\frac{-\Delta t}{k}\right) \\
\end{equation}


%To generate network inputs and outputs distinguished times must be selected and decayed around.
%Each event in the event-stream could be used as a distinguished point and an input/output pair decayed around it.
To generate network input and output images distinguished events must be chosen to be accumulated around.
Every event in the event-stream could be chosen however this would generate a lot of redundant data, as many similar events make up any given line.
Rather an event was selected at uniform sequential intervals to be decayed around, that is, every 150th event would be decayed around. 
Additionally, after the accumulation into frames the order of training example pairs was randomised and split in 70\% training data, 20\% validation data and 10\% test data. 


\subsection{Parameter influence}
%Show some example parameters and how they affect the data \\
A \textit{k} value seemed problem dependent, interesting values to check first were those corresponding to 1, 16, 33 and 100 (\ms). These values were chosen for similarity to conventional frame-based cameras, 16 and 33\ms corresponding to 60 and 30 frames-per-second cameras respectively. 
The values 1 and 100 \ms were chosen to give some idea of how smaller and larger time scales affected network performance. 
A \textit{k} value is considered to be corresponding to a timeslice when at that temporal distance from the distinguished event the accumulation function gives an output of 0.5.
This is a somewhat arbitrary definition as it may instead make more sense to define a corresponding timeslice to be when the accumulation function gives zero (or near zero). 
As the exponential function will only ever approach zero the cut off point will be an arbitrary decision regardless.
In practice using 0.5 was found to perform well as an arbitrary point and meant k values between linear and exponential are comparable.
%TODO include a graphic here demonstrating what corresponding means.
For the linear accumulation function it can be derived that the k values that should be used are given by;

\begin{equation}
    \label{kvalues4linear}
    k = 2 t
\end{equation}

Similarly it can be derived that the \textit{k} values for the exponential case are given by:

\begin{equation}
    \label{kvalues4exp}
    k = \frac{-t}{\ln(0.5)}
\end{equation}

Which when combined result in table \ref{table:kvalues} outlining the choices of \textit{k} to be tested.

\begin{table}[h]
\centering
\begin{tabular}{ | c | c | c | c | c | c |}
    \hline
    t (\ms) &         1 &     16 &    33 &    100  & 165 \\
    Linear &    2 &     32 &    66&     200 & 330\\
    Exp (aprox.)&   1.44 & 23.08 & 47.61 & 144.27 & 238.04\\
    \hline
\end{tabular}
\caption{Values for \textit{k} to be test (in milliseconds)}
\label{table:kvalues}
\end{table}

It is worth noting these values are in \ms while the DVS recordings are in \us so the actual \textit{k} values used are 1000 times larger. 



\section{Sample results}

\begin{figure}[h]
    \centering
    \includegraphics[width=1\textwidth]{varying_accumulation.png}
    \caption{Varying accumulation}
    \label{fig:varyingaccum}
\end{figure}



%It is hard to believe the network will be able to learn any structure from the inputs accumulated over 1 and 16\ms time windows (figures \ref{fig:decay1mslinear}, \ref{fig:decay1msexp}, \ref{fig:decay16mslinear} and \ref{fig:decay16msexp}) in the full 128x128 images.
Using a small time window like 1 or 16 ms may prove a challenging task for the network to learn structure, as these are smaller than the refresh rate of the screen and may be too fine a resolution to contain any motion structure. 
In particular, in the 1 ms accumulation window inputs the signal is represented by only one or two pixels making it indistinguishable from noise in the image. 
In the case of 16 ms the signal from the dot is several pixels but the temporal paths are still much less clear than in the 33 and 100 ms cases.
The 33 and 100 ms accumulation windows allow enough signal to be collected that a clear temporal path can be discerned by a human visual system to derive directional information from just the input image.
This gives some confidence that the frame-based networks will also be able to extract and use this information.

Overall, the exponentially accumulated temporal paths are much more clearly visible in these examples due to the function's aysymptotic nature compared to the linear accumulations hard cut off after a certain temporal difference. 
This also comes at a cost as the exponentially accumulated images also retain noise events for longer.
Whilst the stimulus is present for longer this also allows noise to accumulate. 

These resulting images are blurs through time representing the past and future positions of the dot.
This is in some ways analogous to presenting several sequential frames to a recurrent neural network, in that for each training example it considers not just the current position of the dot (brightest pixels) but also previous positions of the dot (faded pixels).
% Maybe make a comment about how this memory is the fundamental concept around which this work is based
 

\section{Discussion}
The representation of time will have a profound effect on the performance of a network when dealing with data that is rich in temporal information. 
Accumulating events into frames sacrifices some of that temporal information but will allow the use of state-of-the-art learning techniques which are typically frame-based. 
Two different functions (exponential and linear) have been used to accumulate events around a distinguished point to generate training examples for model training. 
Several parameter values were decided upon to give some resemblence to standard frame-based camera recordings rates (being 30 and 60 frames-per-second).
%TODO Rewrite the sentance below
Some values for k result in training examples which are informative to a human visual system and some which are not, which work for the networks examined may likely be a different set. 
The final accumulations were separated into a 70/20/10\% split of training/validation/test datasets for use in training networks later. 

 

        % 4 
\chapter{Pilot studies}
\label{ch:pilot}

%%%%%%%%%%%%%%%%%%%%%%%%%%%%%%      NET 1     %%%%%%%%%%%%%%%%%%%%%%%%%%%%%%%%%%%%%%
\section{pilotNet1}


\subsection{Aims}
The first network created is intended to act as an exploration of the problem space to see what can be learnt and act as a benchmark for later models. 
It was designed to be simple to facilitate reasoning about it's internal dynamics with design decisions specified in table \ref{table:net1def}.


\subsection{Method}
The weights for pilotNet1 were initialised with a truncated normal distribution with a standard deviation of $1 / ( number\_inputs * batch\_size )$ and the biases initialised to zero.
These weight values were chosen to be proportional to the network size and mini-batch sizes. 
Starting the biases at zero was chosen so each unit would consider it's inputs based solely on input weights initially and adjust the biases accordingly in training. 

\begin{table}[h]
\centering
\begin{tabular}{ | l | l | }
    \hline
    Num. Inputs & 16384 \\
    Num. Outputs & 16384 \\
    Connectivity & Fully connected \\
    Num. Hidden Layers & 1 \\
    Size Hidden Layer & 1, 2, 100, 16384  \\
    Activation function & Linear, ReLU, Sigmoid \\
    Loss & Sum Squared Error \\
    Learning rule & S.G.D. (back propogation) \\
    Learning rate & 0.001, 0.1, 0.5 \\
    Mini-batch size & 100 \\
    \hline
\end{tabular}
\caption{Features of net1}
\label{table:net1def}
\end{table}

% TODO discuss S.S.D loss function

%Where each input/output corresponds to a single pixel in the decayed past/future.
Motivation to use one or two units in the hidden layer was derived from the linear nature of the dataset and the thought that the network may be able to model the data with just the gradient of the input.
The network was tested with both linear and non-linear activations to see if a non-linear layer was necesary.

\subsection{Results}
% TODO include some images from Net1 with input, label, output
The performance of the network on a validation set showed a initally rapid decrease follow by very steady decrease suggesting the network has learnt the relatively simple task.
% TODO show sample outputs
However sample outputs from the network were all exactly the same regardless of the input and seemed to be a somewhat random pattern of activations.

\begin{figure}[h]
    \centering
    \includegraphics[width=0.8\textwidth]{pilot_invariant.png}
    \caption{Example Pilot network shifting biases to become input invariant}
    \label{fig:pilotInvariance}
\end{figure}



\subsection{Discussion}
Results from this network highlighted some fundamental properties of the task that were not previously considered as well as some smaller issues with its own design.
Initialisation of the network weights using a standard deviation of $1 / ( number\_inputs * batch\_size )$ was reconsidered as this resulted in unecessarily small weights. 
As the network inputs are in the range [0-1] using a constant standard deviation of 0.1 is expected to give better results. 
Futher, initialisation of the biases to zero could have been causing ties during the back-progation phase and creating odd network dynamics.
To avoid this biases can be initialised just as weights with a normal distribution and standard devition of 0.1.

Training the network with a learning rate of 0.001 showed a smooth decrease in the validation error, increasing the learning rate resulted in a similar curve suggesting 0.001 was smaller than necessary to learn this simple pattern. 

These minor issues with network design were insignificant in comparison to an issue discovered with using the S.S.D. as a loss function. 
%The sample predictions from the network are identical despite different inputs suggesting the network was learning something unexpected.
The network has become input invariant and will predict the same activity pattern for each input example. 
In any given training example most ($>95\%$) of the input vector was zero or near zero meaning when computing the loss for a prediction the network could quickly achieve a small loss by simply outputting zero (or near zero) for every pixel regardless of what the input was. 
Pixels constant in the output correspond to 'hot pixels' in the DVS hardware, given they fire independently of stimuli with their own frequency and location they constitute a sensible prediction for the model. 
%The network achieved this by reducing the biases to the hidden layer isolating the output from the input.
Figure \ref{fig:pilotInvariance} shows that the network achieved this by reducing the hidden layer biases so the hidden layer would output near zero values. 
It simultaneously moved the output biases to near zero meaning the final logits were near also. 
It could then activate just the hot pixels based on each's frequency and keep the others off. 
%The network achieved this by simply reducing its biases with weights staying relatively stable meaning the input signal (the decayed trace of the dot's path) was lost as noise in the system. 


When running the network simulations it was abruptly clear that the network with 16384 hidden units was too large as the Tensorflow computation graph could not fit within the 12GB of memory on a single G.P.U.
The computation graph could be seperated onto multiple G.P.U.s to achieve results however this was considered unnecessary after discovering the problems associated with the loss function and considering the simple nature of the dataset.




%%%%%%%%%%%%%%%%%%%%%%%%%%%%%%      NET 2     %%%%%%%%%%%%%%%%%%%%%%%%%%%%%%%%%%%%%%
\section{pilotNet2}

\subsection{Aims}
% TODO section...

After pilotNet1 some refinements were made although many of the features outlined in table \ref{table:net1def} were kept constant. 
This network is intended to continue the exploratory work on pilotNet1 to see what is immediately possible with a simple network.
The performance of a new loss function which weights mistakes differently is examined as a solution to the input invariance of pilotNet1.

\subsection{Method}
Changes include:
\begin{itemize}
    \itemsep-0.5em
    \item Weights initialised with standard deviation of 0.1
    \item Biases now initialised with normal distribution, standard deviation of 0.1.
    \item Added linear weighting to Loss function.
\end{itemize}

The pilotNet1 loss function suffered from each error the network made being equally weighted.
In any given image the vast majority of pixels should be predicted as near zero, while very few (roughly 20) of the remaining pixels should be active. 
If the network mispredicts an \textit{on} pixel (that is a pixel which should be near one), this is a much more serious issue than if the network was to mispredict an \textit{off} pixel (a pixel that should be near zero). 
To minimise this issue a penalty could be applied to each type of mistake weighing incorrect active pixels (i.e. predicting one instead of zero) as only a small error while mispredicting an inactive pixel is considered a serious error.
Essentially penalising the network heavily for failing to predict the path of decayed pixels. 

\subsubsection{Weighted prediction penalty}
The weighting of penalty should be proportional to the activity in the input, such that if the signal is weak then any mispredictions should be penalised heavily compared with a strong signal where a few mistakes make less difference. 
% TODO Tidy up this maths and maybe take it out of line
The activity of a given image is defined as the sum of the activity of each pixel within the image.
If the activity is given by some variable $g$ then the penalty weighting for mispredicting a pixel as \textit{off} when it should be \textit{on} should be $(16384 - g) / 16384$, similarly a misprediction of a pixel as \textit{on} when it should be \textit{off} should only be $g / 16384$. 
In a system where neuron outputs were binary this would be all, however in the continuous output demmanded by a decayed representation inbetween values must be considered.
% TODO Add in linear equation and reference below
Interpolating linearly between these two points gives Equation \ref{eq:weightedPenalty} which can be used to calculate the loss weighting for a given pixel, $p$.
Combining this weighting with the normal S.S.D. error function gives Equation \ref{eq:linearLoss} which describes the loss for each pixel. 
%In a binary system this would simply be a matter of weighing each  equation used is given in *** Ref equation *** 

% TODO Equation goes here
\begin{equation}
    \label{eq:weightedPenalty}
    W(p) = \frac{16384 - 2g}{16384} p + \frac{g}{16384}
\end{equation}

Given a prediction pixel, $pred$, the linearly weighted loss for that pixel is:

\begin{equation}
    \label{eq:linearLoss}
    L(pred) = (label - pred)^2 W(label)
\end{equation}

The final loss for a whole training example is then the sum of the loss for each individual pixel. 

\subsection{Results} 
Similarly to the pilotNet1 results this network also quickly learnt to be input invariant by shifting the biases to cause the output to be constant near zero. 
An example of the constant output is shown in \fig{fig:pilotInvarianceSample}. 
This invariance was not affected by the number of hidden units used

\begin{figure}[h]
    \centering
    \includegraphics[width=0.8\textwidth]{pilotInvariantSample.png}
    \caption{Training pairs in the PilotNets with constant predictions}
    \label{fig:pilotInvarianceSample}
\end{figure}


\subsection{Discussion}
Despite the various improvements offered over pilotNet1 this network still suffers from shifting weights and biases to generate a constant output regardless of the input. 
The signal to noise ratio of roughly $20/16384$ is to great to overcome with a simple linear weighting as specified above.
Perhaps the same tests with a larger dot which causes a greater number of events might be able to learn the pattern. 
Such an experiment is in some ways analogous to the work done in chapter \ref{ch:attentional}. 
Alternatively if the network could ignore large sections of the input and just search in smaller patches it may be able to extract some useful information.
This will be explored further in chapter \ref{ch:convolutions}. 

% TODO Consider moving to final discussion
The pilot study while unsuccessful in itself revealed interesting insight into the nature of the problem, primarily the signal to noise ratio and helped inform other network design decisions.
 
             % 4
\chapter{Study 1 - Convolutional architectures}
\label{ch:convolutions}

%%%%%%%%%%%%%%%%%%%%%%%%%%%%%%      EVO Kernels    %%%%%%%%%%%%%%%%%%%%%%%%%%%%%%%%%%%%%%
\section{Evolutionary kernels}

\subsection{Aims}
%Tried to use evol kerns but sparse nature of data means no good
After the results from the Pilot study it was clear the task needed to be reframed.
Previous work using convolutions and kernels as feature detectors suggested they might be able to provide feature maps necessary for a network to learn.
%Additonally using convolutions would have the advantage that they are able to ignore much of the image and focus on areas of activity.  
Convolutions may be well suited to this problem as they naturally focus on only small segments of the image meaning the signal-to-noise ratio affecting the PilotNets may be less problematic. 
Kernels capable of detecting dot motion were developed using an evolutionary algorithm. 
These dataset sepecific kernels were then used to transform the DVS recordings into feature maps in which each pixel represented the kernel for which it most highly responded. 
The feature maps were then used as training examples in a fully connected network with the aim being to analyse the performance of the network in predicting an output feature map. 
%Kenels specialised to the datasets were developed, these were then used to preprocess the input/output decayed images to produce feature maps which could then be used to train the network as per normal.
%Kernels  to the datsets were developed and then used to process recordings into training examples.

\begin{figure}[h]
    \centering
    \includegraphics[width=0.8\textwidth]{evoNetStructure.png}
    \caption{Structure of the Evolutionary Kernels processing pipeline}
    \label{fig:evoNetStructure}
\end{figure}

\subsection{Method}
Three major steps were required to convert from a DVS recording to a prediction in this study; the final system is illustrated in figure \ref{fig:evoNetStructure}.
First kernels for convolutions were evolved to be specialised to the 8AD dataset using a 1 + 1 hillclimbing algorithm (described in Appendix \ref{ch:evolution}). 
After sufficient evolution they were convolved with the DVS recording to produce feature map training examples.
The feature maps specifying which kernel had responded most strongly for that pixel were fed into a fully connected network which predicted future feature maps at the output.  

\begin{figure}[h]
    \centering
    \includegraphics[width=0.7\textwidth]{evoStableTrimmed.png}
    \caption{An example kernel evolution converging to a stable state}
    \label{fig:evoStable}
\end{figure}

\subsubsection{Evolving kernels}
As no standard set of feature kernels to use with event-based data exists these would need to be created.
Previous work developing kernels using an evolutionary algorithm made this a sensible place to start.
In this work a kernel is considered as a matrix with each value being a weight describing how important an event at that position is.
Kernels start randomly initialised and are iteratively updated by permuting kernel weights and convolving the new kernel with some sample data.
Improvements in kernel performance as measured by a fitness function are kept for the next step of evolution.
Figure \ref{fig:evoStable} shows what a kernel evolving from an initially random state (top row) looks like after 200 evolutionary steps (bottom row).
Each row in the image represents the kernels state at an evolution step while each column represents an individual pixel's value at any step. 
A stable kernel signifies the kernel has found a local maximum in capturing information from a particular recording.
Finer details of the evolutionary algorithm are discussed in Appendix \ref{ch:evolution}.
A set of nine and a set of five kernels were created using this technique.
Motivation for using nine kernels was inspired by the 8AD dataset with anticipation that each kernel would specialise for one of the angles plus one kernel to detect noise.
Five was chosen to see if similar behaviour could be modelled as weighted combinations of less kernels.
A kernel size of 11x11 pixels was chosen as this would capture much of the temporal past (and future) for an accumulation over 33\ms.



\subsubsection{Processing training examples}
Convolving the 9, 11x11 kernels with the 128x128 images gave 9, 128x128 feature maps.
Rather than using pooling \textit{within} a feature map as is standard in convolutional neural networks, max-pooling was applied \textit{between} the maps. 
The result was a single 128x128 map created in which each position was the index of the feature map with the highest output at that pixel.
An output feature map then represents which kernel the network predicts will be most active in any given pixel.
Using the output feature map in combination with the kernels a future temporal surface could then be reconstructed. 
%This decision was motivated by the idea that if the network learnt which kernels map to which a temporal surface prediction could be reconstructed using the output feature map with the kernels. 


\subsubsection{Network design} 
The networks used then resembled those of PilotNet2.
Representing each pixel as the kernel which most strongly responds to it should make predicting future motion a simple task for a shallow network to learn.

\begin{figure}
    \centering
    \includegraphics[width=0.75\textwidth]{anaKerns.png}
    \caption{Example of four analytic kernels used}
    \label{fig:anaKerns}
\end{figure}

\subsubsection{Analytic kernels}
Additionally an alternative set of kernels was developed based on the probabilities of event patterns in the training data.
For all samples of a given angle an 11x11 matrix was cropped around each event; the probabilities of events in each position of the matrix was then calculated for that angle giving an analytic kernel.
The process of cropping an 11x11 matrix around events is the same process as discussed in Chapter \ref{ch:attentional}.
These kernels were significantly quicker to compute compared to the evolutionary kernels which required many convolutions of the full dataset for each evolution. 


\begin{figure}[h]
    \centering
    \includegraphics[width=0.7\textwidth]{evoUnstableInsert.png}
    \caption{(a) Non-converged evolution history for the final kernel shown in (b)}
    \label{fig:evoUnstableInsert}
\end{figure}

\subsection{Results}
Using the available evolutionary algorithm proved to be too slow to develop meaningful 11x11 kernels on the large 8AD dataset.
The kernels were not able to converge to a stable point after 14 days of training; example kernels are shown in \ref{fig:evoUnstableInsert}.
Kernels showed signs of moving towards appropriate features (e.g. higher activations along the South-West diagonal) but still needed more time to develop. 
The kernel states after 14 days were applied to the data regardless to see if meaningful results could be achieved. 
Kernels were evolved with integer values for weights but before use the values were mapped to the range 0 to 1.
After application to the data the network became input invariant as per the pilotNets. 
The analytic kernels, shown in figure \ref{fig:anaKerns}, were also substituted in place of the evolved kernels and the processing pipeline re-run to get the same input invariant results. 


\subsection{Discussion}
Developing kernels evolutionarily which would be specialised to the dataset proved to be too computationally expensive for the duration of this project. 
A faster algorithm could have been implemented but was considered out of scope meaning only premature kernels were used.
Analytic kernels offered a promising alternative to evolutionary algorithms, being much faster to compute and resembling the 8 angles clearly.
However, the analytic kernels were also unable to produce meaningful results suggesting a deeper problem remains. 
The deeper problem may be that the network which is still taking all 16384 inputs is suffering from a signal-to-noise-ratio problem.

Using inter-layer pooling to generate a final feature map which has map indices from the previous layer as values may need to be reconsidered.
In this scenario there is no meaningful reason why the kernels should be ordered, yet they will be given order by the process of using their indices.
Hypothetically if the North-East diagonal was index five, South-West was six and South-East was seven, then the network would need some way of rationalising what a value of 5.5 or 6.5 meant. 
Ordering the kernels based on angle might be a start to solve this problem but does not leave room for a noise kernel.
Further, this solution does not generalise to other more complex problem sets.
Methods of encoding the kernels as one-hot vectors were investigated but no functional method was found. 
%Using feature maps as training examples for a fully connected network did not provide a suitable solution to produce meaningful predictions. 
%The networks consistantly learnt to ignore the input, shift biases and output zeros to minimise loss quickly. 


%%%%%%%%%%%%%%%%%%%%%%%%%%%%%%      CONVOLUTIONAL NN    %%%%%%%%%%%%%%%%%%%%%%%%%%%%%%%%%%%%%%
\section{convNet}
%Define it what did it learn and what happened.
%Alternative approach of using evol kerns on full image

\subsection{Aims}
The amount of time required to evolve kernels using the available evolutionary algorithm meant an alternative approach was necessary. 
Directly using a conventional convolutional network represented the logical progression after evolving kernels manually. 
The experimental design remained similar to the evolutionary kernels as described in figure \ref{fig:evoNetStructure}.
The difference being for this study the 128x128 temporally accumulated images were directly used as inputs to the network which was responsible for developing the kernels and feature maps and pooling was done spatially within a feature map rather than between feature maps.
This study should be able to leverage the advantages of convolution networks (e.g. focused feature detectors on smaller parts of the image) whilst still being trainable in reasonable time. 

\begin{table}[h]
\centering
\begin{tabular}{ | l | l | }
    \hline
    Num. Inputs & 16384 \\
    Num. Outputs & 16384 \\
    Num. Hidden Layers & 3 \\
    Fully connected & 64, 1024 units \\
    Layers & Convolutions -\textgreater pooling -\textgreater output \\
    Output Activations & Linear, Sigmoid, ReLU \\
    Loss & Sum Squared Difference, weighted S.S.D. \\
    Learning rule & S.G.D. (back propogation) \\
    Learning rate & 0.5 \\
    Convolution stride & 1 \\
    \hline
\end{tabular}
\caption{Features of convNet}
\label{tb:convNetdef}
\end{table}

%Rather than using an evolutionary algorithm to evolve kernels with which to later apply convolutions to the data this can all be done within a Convolutional Neural Network. 
%This network should be able to produce much of the behaviour of the evolved kernels (because it will be designing its own during training) but shouldn't suffer from the slow training time and possible inconsitancys between training and test data.

\subsection{Method}
\label{sec:convMethod}
The network used in this work consisted of an input layer followed by a convolution layer with 9, 11x11 convolutions followed by a 2x2 max pooling layer feeding into a fully connected layer. 
The output layer of the network was differently activated to compare performance. 
Details are outlined in table \ref{tb:convNetdef}.


\subsection{Results}
The convolution networks quickly (\textless 250 epochs) become input invariant.
This is shown in figure \ref{fig:convInputInvariance} in which despite a strong, relatively clean signal the network is not able to make a meaningful prediction. 
The mechanism for this invariance can be seen in the weights and biases of the network over the first 5000 epochs.
The weights shift between 3 and -3 while the biases quickly shift towards -3, this example network (ReLU activated) then outputs just zeros.
This trend was consistent across the other convolutional architectures and parameters outlined in table \ref{tb:convNetdef}.

\begin{figure}[h]
    \centering
    \includegraphics[width=0.8\textwidth]{convNetInvariance.png}
    \caption{An example of a convolutional network input invariance and the corresponding fully connected layer features}
    \label{fig:convInputInvariance}
\end{figure}

\subsection{Discussion}
Convolutional architectures should have been well suited to solving this sparse signal problem.
The architectures used in this study were able to make any meaningful predictions and quickly learnt to just output a constant zero pattern much like the PilotNets.
Several factors could have contributed to the poor performance of these networks including signal-to-noise ratio, number of epochs, learning rate or the training data. 

The convolutional networks were believed to have been better able to deal with the signal-to-noise ratio, however the ratio may have proved still too high for the networks.
Convolution sizes were chosen to consider this with an 11x11 convolution reasonably conveying an accumulated past/future over a 33 ms window. 
An 11x11 area may have been too large yet the 6x6 kernels were not able to learn either, suggesting there may still be other problems.
The noise in the training data may have proved problematic for learning. 
Each kernel may have specialised to pick up different kinds of noise and as noise is inherently unpredictable the kernels learn to predict zero. 

Networks were trained with 50,000 epochs, which when compared with how quickly the network became input invariant was considered sufficient.
This number of epochs may not have been enough, perhaps the network needed more time to fine-tune the weights. 
The need for fine-tuning would be unlikely given the lack of improvement in the first 50,000 epochs. 
The learning rate may have contributed to the network getting stuck in a local minima (outputting just zeros) and trying alternative learning rates may help the network learn.
Other techniques such a momentum \ref{sutskever2013importance} which can assist in learning may help but are left as future work.


% - Why did it perform so badly?? -> signal to noise, but even attentional convolutional fail later... Something inherent to convolutions?
 %- Was 1024 Units too many / not enough?
% - Were kernels comparable to evolved kernels / Analytic kernels?
% - How could this be improved?
% - Was the spacial max-pooling an influence




























      % 4
\chapter{Attentional Networks}
\label{ch:attentional}
To help simplify the learning problem the task was again reframed to improve the signal to noise ratio.
The preprocessing was adjusted such that decay was applied around every event but instead of using the full image only an 11x11 decayed area was kept.
In keeping with previous experiments decay was applied into the past and future to be the input and output. 
Figure \ref{fig:11inoutpair} is an example of such a pair.
%Rather than applying decay to the whole image at uniform time intervals and using that as input the to the network each event was decayed around resulting in many more (albiet similar) training examples.
%However only an 11x11 area around each event was considered meaning the signal to noise ratio was much higher. 

\begin{figure}[h]
    \centering
    \includegraphics[width=0.5\textwidth]{11xinoutpair_83.png}
    \caption{Example of 11x11 input for an attentional network}
    \label{fig:11inoutpair}
\end{figure}

Figure \ref{fig:11inoutpair} is a cherry picked training example though.
The Attentional networks still have a noisy task to solve because all events, including noise events, are considered. 
Many of the training examples derived from noise pixels could be filtered out efficiently by demmanding the total activity in a training example excede some low threshold.
However, the system should be able to deal with noise and setting such a threshold would create another unnecessary hyper-parameter to the model.
Additonally such a parameter would be dependent on the time scale of the data and would need to be adjusted for each task. 
As will be shown the network is able to learn even in the midst of such noise so it is left in. 

%%%%%%%%%%%%%%%%%%%%%%%%%%%%%%      ATTENTIONAL NN    %%%%%%%%%%%%%%%%%%%%%%%%%%%%%%%%%%%%%%
\section{Attentional Directly Connected (ADC) Network}
This network, as the name suggests, is a direct connection between the input and output units.
The prediction problem had been broken down into a seemingly simple task so it was expected that results could be achieved with a simple network. 
The network details are outlined in table \ref{tb:attnet1def}, with key points being the loss function has returned to the standard S.S.D. instead of the linearly weighted S.S.D. and the number of inputs has drastically decreased. 

\begin{table}[h]
\centering
\begin{tabular}{ | l | l | }
    \hline
    Num. Inputs & 121 \\
    Num. Outputs & 121 \\
    Connectivity & Fully connected \\
    Num. Hidden Layers & 0 \\
    Activation function & Linear, ReLU, Sigmoid \\
    Loss & Sum of Squares Difference \\
    Learning rule & S.G.D. (back propogation) \\
    Learning rate & 0.1 \\
    Mini-batch size & 100 \\
    \hline
\end{tabular}
\caption{Features of the Attentional Directly Connected Networks}
\label{tb:attnet1def}
\end{table}

\subsection{Linear activation}
The simplest ADC network considered had only a linear activation to compute outputs. 
This proved to be enough for the network to learn to make coherent predictions. 
The two datasets (8 Angle and Arbitrary Angle) were considered, a network was trained on each and then made predictions on a set of validation data from it's own dataset and other dataset to see how it could generalise.
For clarity the network trained on the 8AD will be called 8AngNet and the network trained on the AAD will be called ArbAngNet. 

\subsubsection{8AngNet results}

\begin{figure}
    \centering
    \includegraphics[width=0.8\textwidth]{ADC_8a_8a_13.png}
    \caption{A good prediction from 8AngNet on the 8AD validation set.}
    \label{fig:ADC_8a_8a_crct} 
\end{figure}

\begin{figure}
    \centering
    \includegraphics[width=0.8\textwidth]{ADC_8a_8a_7.png}
    \caption{A prediction from 8AngNet with a noisy input}
    \label{fig:ADC_8a_8a_noisy}
\end{figure}

Figures \ref{fig:ADC_8a_8a_crct} and \ref{fig:ADC_8a_8a_noisy} show how the linearly activated 8AngNet predicted in two cases from the 8 Angle validation set.
This prediction looks promising that the network is capable of representing some structure of the data as the prediction is similar to the label (ignoring some noise).
% TODO Make this an even more correct image and use this particular img (13) later to discuss problems
%In figure \ref{fig:ADC_8a_8a_crct} the network is performing well and gives a prediction which is quite similar the ground truth (ignoring noise). 
Figure \ref{fig:ADC_8a_8a_noisy} shows a noisy training example.
The label does not intuitively follow from the input and the network only outputs small values.
However the network does predict faintly along the top left diagonal which is sensible when considering the input has two pixels along the bottom right diagonal (the closer of which is highly active making it resemble a decayed path). 
%However the network has noticed two pixels active along the bottom right diagonal (one of which is highly active) and the network predicts a faint output along the top left diagonal.

\begin{figure}
    \centering
    \includegraphics[width=0.8\textwidth]{ADC_8a_aa_4.png}
    \caption{8AngNet struggles to predict inbetween paths}
    \label{fig:ADC_8aNoaa}
\end{figure}

\begin{figure}
    \centering
    \includegraphics[width=0.8\textwidth]{ADC_8a_aa_46.png}
    \caption{8AngNet predicting a slightly off angle input}
    \label{fig:ADC_8aNoaa_fork}
\end{figure}

The network performing well on its own validation set is a success in itself but raises the question of how it will generalise from 8AD to AAD.
It was hypothesised the network might be able to use a combination of known angles to represent the new angles in the Arbitrary Angle dataset.
This was not the case though, figure \ref{fig:ADC_8aNoaa} shows the network stuggling to predict the motion.
Most of the activity in the prediction falls in the East North-East section which is where the input was.
There is some very limited activity that matches the label but this is insignificant compared to previous predictions and what can be realistically expected from the network. 
Further figure \ref{fig:ADC_8aNoaa_fork} shows a slightly off center input which resembles an angle from 8AD. 
The network has trouble interpreting this and makes 3 very faint predictions being the top right diagonal, to the right edge and along the input.
This suggests the network is not able to efficiently represent the arbitrary angles as some combination of the angles it learnt and could perform reasonably well on.

An additional interesting case which further supports this claim is seen in figure \ref{fig:ADC_8aNoaa_special} which shows the network suffering from some neatly aligned noise. 
The network is well equipped to deal with inputs coming from one angle at a time but this noise makes it appear as if two dots may be crossing paths. 
The networks behaviour to predict two strong output paths shows that each input path (and its predictions) are happening independently of the rest of the input.
If the network was representing the input as an angle it would be reasonable to expect that the output might be a blur in the top right corner of the prediction.  
Instead this clear prediction of two paths suggests the network is simply learning to map between areas of the input to areas of the output. 

\begin{figure}
    \centering
    \includegraphics[width=0.8\textwidth]{ADC_8a_aa_15.png}
    \caption{8AngNet network predicting two paths due to noise}
    \label{fig:ADC_8aNoaa_special}
\end{figure}


\subsubsection{ArbAngNet results}
It follows that a network trained on only 8 angles might have trouble generalising to arbitrary angles so a second network was trained on the Arbitrary Angles Dataset. 
In general the network trained on Arbitrary Angle data was less confident in its predictions (magnitude of predictions were lower) but in each guess it would cover a boarder area. 
% TODO rewrite this sentence
An example of this is figure \ref{fig:ADC_aaaa_crct} in which the prediction has many faintly coloured squares, in contrast to the more confident predictions made in figure \ref{fig:ADC_8a_8a_crct} by 8AngNet.

\begin{figure}[h]
    \centering
    \includegraphics[width=0.8\textwidth]{ADC_aa_aa_4.png}
    \caption{ArbAngNet correctly predicting}
    \label{fig:ADC_aaaa_crct}
\end{figure}
% TODO these visualisations can be improved by grouping the predictions from both nets into 1 img


\begin{figure}[h]
    \centering
    \includegraphics[width=0.8\textwidth]{ADC_aa_aa_15.png}
    \caption{ArbAngNet predicting two paths due to noise}
    \label{fig:ADC_aaaa_twopath}
\end{figure}

\begin{figure}[h]
    \centering
    \includegraphics[width=0.8\textwidth]{ADC_aa_aa_46.png}
    \caption{ArbAngNet predicting a slightly off angle input}
    \label{fig:ADC_aaaa_fork}
\end{figure}


The neat noise example which suggested that 8AngNet was simply mapping regions in the input to regions in the output shows a very different prediction from ArbAngNet.
ArbAngNet does not predict any given angle strongly but instead has a very faint prediction along both but also between the two lines. 
The significance of these prediction is questionable given small the predictions are but it does give some insight into the networks dynamics.
It should be noticed that this does not exclude the possibility that ArbAngNet is also just mapping between regions of the input and output, rather this is still a promising theory.


%%%%%%%%%%%%%%%%%%%%%%%%%%%%%%      ATTENTIONAL NN    %%%%%%%%%%%%%%%%%%%%%%%%%%%%%%%%%%%%%%
\section{Attentional Hidden Layer (AHL) Network}

Discuss results of using a hidden layer when done...


%%%%%%%%%%%%%%%%%%%%%%%%%%%%%%      AUTO ENCODE     %%%%%%%%%%%%%%%%%%%%%%%%%%%%%%%%%%%%%%
\section{Auto Encoder}
How much could be cleaned up \\
Retry other networks with cleaned up data -- How do they go? \\

       % 8
%\chapter{Neural Networks}

To process the data several different network architectures were examimed. 
Unless otherwise specified networks were run with a mini-batch size of 100, trained using Stochastic Gradient Descent (S.G.D.) with backpropogation and run for 1 million epochs. 
Stochastic Gradient Descent (S.G.D.) was used to facilitate quick learning as more complex methods were deemed unnecesary in such a simple task. 
Weights and biases were initialised using a truncated normal distribution with a standard deviation of 0.1 unless otherwise specified.
Detailed Tensorboard representations of each network can be found in Appendix \ref{ch:tbnets}.

%%%%%%%%%%%%%%%%%%%%%%%%%%%%%%      NET 1     %%%%%%%%%%%%%%%%%%%%%%%%%%%%%%%%%%%%%%
\section{Net1}
The first network created was simple by design to act as a benchmark against which other models could be compared. 
It was defined as outlined in table \ref{table:net1def}.
The weights for Net1 were initialised with a truncated normal distribution with a standard deviation of $1 / ( number\_inputs * batch\_size )$ and the biases initialised to zero.

\begin{table}[h]
\centering
\begin{tabular}{ | l | l | }
    \hline
    Num. Inputs & 16384 \\
    Num. Outputs & 16384 \\
    Connectivity & Fully connected \\
    Num. Hidden Layers & 1 \\
    Size Hidden Layer & 1, 2, 100, 16384  \\
    Activation function & Linear, ReLU, Sigmoid \\
    Loss & Sum Squared Error \\
    Learning rule & S.G.D. (back propogation) \\
    Learning rate & 0.1, 0.5 \\
    \hline
\end{tabular}
\caption{Features of net1}
\label{table:net1def}
\end{table}

Where each input/output corresponds to a single pixel in the decayed past/future. 
Motivation to use one or two units in the hidden layer was derived from the linear nature of the dataset and the thought that the network may be able to model the data with just the gradient of the input.
The network was tested with both linear and non-linear activations to see if a non-linear layer was necesary. 


%%%%%%%%%%%%%%%%%%%%%%%%%%%%%%      NET 2     %%%%%%%%%%%%%%%%%%%%%%%%%%%%%%%%%%%%%%
\section{Net2}
After Net1 some refinements were made although many of the features outlined in table \ref{table:net1def} are kept constant with Net2. 
Some changes include:
% TODO Tidy up this list, the gaps are too large...
\begin{itemize}
    \item Weights initialised with standard deviation of 0.1
    \item Biases now initialised with truncated normal distribution, standard deviation of 0.1. 
    \item Added linear weighting to Loss function.
\end{itemize}

%%%%%%%%%%%%%%%%%%%%%%%%%%%%%%      EVO Kernels    %%%%%%%%%%%%%%%%%%%%%%%%%%%%%%%%%%%%%%
\section{Evolutionary kernels}
%Tried to use evol kerns but sparse nature of data means no good
After the results from Net1 and Net2 (discussed in sections \ref{sec:net1discuss} and \ref{sec:net2discuss}) it was clear the task needed to be reframed. 
Previous work using convolution and kernels as feature detectors suggusted they might be able to provide the feature maps necessary for a network to learn. 
First kenels specialised to the datasets would be developed, these could then be used to preprocess the input/output decayed images to produce informative feature maps which could then be used to train the network as per normal. 

%TODO Need a clear definition of what the kernels are 
% Things like:
%   - Size
%   - Limits (27) why...

As no standard set of kernels to use with event-based data exists these would need to be created.
Previous work developing kernels using an evolutionary algorithm made this a sensible place to start.
Kernels start randomly initialised and are iteratively updated by permuting kernel weights, improvements as measured by a fitness function are kept.
Finer details of the evolutionary algorithm are discussed in Appendix \ref{ch:evolution}.

% TODO This part of the research...
A set of nine and a set of five kernels were created using this technique.
Motivation for using nine kernels was inspired by the simple 8 angle dataset with the hope each kernel would specialise for one of the angles plus one kernel to detect the noise. 

Five was chosen to see if similar behaviour could be modelled as a combination of less kernels. 
Convolving the 9, 11x11 kernels with the 128x128 images gives 9, 128x128 features maps.
Rather than using pooling as is standard in convolutional neural networks a single 128x128 map was created in which each position was the index of the feature map with the highest output at that pixel. 
% TODO comment on how ties were broken. or just remove this whole section..........


MORE TO DO HERE...
Discuss how kernels are only applied around event
meaning 11x11x1
What exactly is that 1...

The network used then looked much like Net1 and Net2. 


%%%%%%%%%%%%%%%%%%%%%%%%%%%%%%      CONVOLUTIONAL NN    %%%%%%%%%%%%%%%%%%%%%%%%%%%%%%%%%%%%%%
\section{convNet}
%Define it what did it learn and what happened.
%Alternative approach of using evol kerns on full image

Following on from evolving kernels manually are convolutional neural networks.
The network used in this work consisted of an input layer, an 9, 11x11 convolutions followed by a 2x2 max pooling layer feeding into a 1024 unit sigmoid layer which fed into a 16384 linear output layer. 





\pagebreak
%%%%%%%%%%%%%%%%%%%%%%%%%%%%%%      ATTENTIONAL NN    %%%%%%%%%%%%%%%%%%%%%%%%%%%%%%%%%%%%%%
\section{Attentional Networks}

In an attempt to provide a better signal to noise ratio to the network an attentional 



%%%%%%%%%%%%%%%%%%%%%%%%%%%%%%      AUTO ENCODE     %%%%%%%%%%%%%%%%%%%%%%%%%%%%%%%%%%%%%%
\section{Auto Encoder}
How much could be cleaned up
Retry other networks with cleaned up data. 

















       
%\chapter{Discussion}

\ldots\ or perhaps the discussion should be a separate chapter.

In any case, you will probably need to include tabulated results.
\tab{tf2} illustrates the use of various \LaTeX\ environments to
include a computer printout (plain text file) in a document.  The
\texttt{verbatim} environment, which encloses the formatted text, is
also useful for program listings.

        
\chapter{Conclusions}

\section{Findings}

\section{Implications}

\section{Possible future work}

       % 4 



%%%%%%%%%%%%%%%     REFERENCES    %%%%%%%%%%%%%%%%%
%\begin{thebibliography}{99}
%\addcontentsline{toc}{chapter}{References}
%\bibitem{lamport} L.~Lamport, \emph{\LaTeX: A Document Preparation
%System}, 2nd ed. (Addison-Wesley, 1994).
%\bibitem{LABEL2} REFERENCE 2
%\bibitem{ETC.} Etc.
%\end{thebibliography}

\bibliographystyle{IEEEtran}
\bibliography{/home/joti/Documents/References/mendeley_bibs/thesis}



%%%%%%%%%%%%%%%%%    APPENDIX    %%%%%%%%%%%%
\appendix
% Chapters after the \appendix command are lettered, not numbered.
\newpage
\mbox{}
\newpage

% \include appendix chapters here.

\chapter{Evolutionary Kernels}
\label{ch:evolution}

Content about the evolutionary kernels

\chapter{Tensorboard network structures}
\label{ch:tbnets}

This appendix will hold tensorboard images of each network for reference
\section{Net1}
\section{Net2}
\section{Evolutionary Kernels network}
\section{convNet}
\section{Attentional Network}
\section{AutoEncoder}


\cleardoublepage


\end{document}
