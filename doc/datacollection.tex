\chapter{Datacollection and pre-processing}

\section{The need for a dataset}
To fairly compare models it is necessary for the community to have standardised datasets (REF MNIST, CIFAR ETC.). % Reference MNIST, CIFAR etc.
Many datasets exist for time-stepped models from well formatted and normalised datasets (such as MNIST) to large complex datasets (such as ***ImageNet dataset?***).
In an attempt to compare event-based processing models to time-stepped models some have tried using DVS recordings of static MNIST digits (REF O'CONOR). % ref OConor
However this seems to defeat the purpose of the DVS in its ability to capture spatio-temporal surfaces if the surfaces are inherintly 2D.
Newer benchmark datasets have addressed this problem offering various stimuli, in particular complex real world self motion for a mobile robot (REF AMY PAPER). %% Ref Amy
There is still a deficency of extremely simple datasets around which strong mathmatical foundations can be laid to allow a deep understanding of network dynamics. 
As movement is a key advantage of the DVS over time-stepped models this should be emphasised in the dataset. 
%This work offers such a dataset consisting of dots moving linearly with constant velocity. 

\section{Requirements and details}
The reason behind such a dataset must be carefully considered so when collected it forms a comprehensive set.
Simple structure is required so that when experimenting with new appraoches to processing any dynamics in the system can be reasoned about with some accuracy.
Additionally a dataset for complex real world (robotic self-motion) or interesting patterns already exists, it is simpler data that is necessary.
To evaluate the accuracy of classification systems a ground truth is also required labelling events as signal or noise. 
With the advent of deep learning the advantages of large bodies of training data have become apparent. 
Thus the ability to collect a large number of samples is also deemed necessary and the system should be as autonomous as possible. 

Considering the gap this dataset will fill and the requirements specified the simplest datum is a single dot moving linearly with a constant velocity. 
The variables to be manipulated include the dot diameter, velocity, direction, the variations recorded are listed in *** REF TABLE ***.
As a stepping stone from simple datasets to more complex real world datasets two further variations were considered; two dots intersecting at the center of the screen and multiple dots tracing arbitrary angles (occationally intersecting). 
The prediction as two (or more) dots intersect will reveal insight into how the systems inner workings and is considered worth including. 


%\begin{tabular}{ c | c | c }
%  Dot size & 4 & 6 & 8 \\
%  4 & 5 & 6 \\
%  7 & 8 & 9 \\
%\end{tabular}

\section{Considerations}
With clear requirements the practical generation of the dataset can be considered in detail.
The requirement for constant velocity would prove difficult to control in any real world system with forces of friction and gravity.
Such a real world system would also prove uneconomical in terms of time required to construct and unless well automated also in terms of ease of generating recording data. 
Recording computer simulations becomes an attracive option but comes with its own negatives including screen refresh rates and non-continuous stimuli. 
 

Flashes on screen -> noise
Segmenting datasets
Line angles (Close to edge, just a tiny segment in corner)
Ground truth to actual recording consideration (mitigated through setup?)
Light invariance
Black on white and white on black, grey
Hot pixels

\section{Methodology}
What did i actually end up doing

\section{Final dataset specifications}


\section{Comments}
Looking back at the data what do i think?
%Flat timestamps?? -> experimental hardware
Segmenting looses a lot on each side
No metric for validating ground truth against actual dot accuracy
Camera angles (parallel lines not so).




