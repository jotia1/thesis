\chapter{Introduction}

%The introductory chapter describes the importance of the field and the
%scope and significance of your project.  It usually ends with an
%overview of the remainder of the thesis.

%Notice that Arabic page numbering begins with Chapter 1.  Preceding
%pages (known as ``frontmatter'') have Roman numbering.  The
%\texttt{book} document class in \LaTeX\ follows this numbering
%convention by default (see Lamport~\cite{lamport}, p.\,80).





%% Broard intro to topic, where did this come from, what is the problem, 
Vision is the primary sense used by many natural agents to gather information about their environment.
It would seem to follow that vision should be an important percept for artificial agents.
%It would seem to follow that artificial agents should also be able to leverage visual sensors in a similar way.
% TODO Is digital systems the right term to use here??
However the apparent ease with which natural systems process visual information does not translate to standard digital arcitectures, making vision a seldom used sense. 
In real time applications vision is often impractical due to the computational load required to extract meaningful information. 
%In its place alternatives such as Infra-red for distance sensing or L.I.D.A.R. for mapping are substituted.
% TODO gonna need a refernce below
In a more theoretical environment there are still many challenges in extracting meaningful information from standard vision data due to variations in lighting, orientation, position, scale etc.


%% Breif into to neural nets
%   Why they are better than hard coded rules -> they can learn
A simplistic approach to vision processing would entail defining exact features of the object to be recognised.
This would quickly become overwhelming though as the system decends into considering a seemlingly endless list of special cases and permuations that a single object could take in an image.
% TODO will need a reference to standad analysis algoithms
%Numerous algorithms and heuristics have emerged making meaningful analysis possible in some circumstances such as ****** Color filtering / normal face detection?? / Canny edges? *******.
%Unfortunately these struggle to generalise to arbitrary object classification or prediction.
% TODO reference some state-of-the-art networks
%A promising alternative, Neural Networks, currently have state-of-the-art performance on many of the public benchmark datasets.
Alternatively an algorithm designed to detect a specific object using heuristics or traditional vision processing methods could be used.
Unfortuneately these only solve a smaller part of the processing problem, such as canny edge detection, or struggle to generalise to objects other than that they were designed for, such as ** face detection **.
Neural networks offer a general solution to the problem of classification and prediction.
Rather than relying on expertly designed heuristics or algorithms a neural network is randomly initialised and after repeated presentations of some stimuli can adjust internal parameters to minimise a loss function. 
This ability to self adjust and decide which parts of the stimuli are important has lead to systems capable of learning representations far more complex than an expert would have been able to design explicitly. 


Breif into to Event-based sensors \hfill 
   fast, low power, sparse, biologically realistic \hfill

Why the DVS with NN's makes sense \hfill 


%\section{}
