\chapter{Literature - Sensors, Processing, Networks}

%%%%%%%%%%%%%%%%%%%%%%%%%%    ASYNC SENSORS     %%%%%%%%%%%%%%%%%%%%%%%%%%%%%%%%%

\section{Asynchronous Vision Sensors}  % 1 page
Lots of different types of dynamic vision sensors \cite{delbruck2010activity}
Why would we want a DVS? microparticle tracking \cite{ni2012asynchronous}

\subsection{Dynamic Vision Sensor (DVS)}
Tobi made a camera \cite{delbruck2008}
Discuss the Dynamic vision sensors, what they are capable of
Used in stuff like \cite{delbruck2007fast}
They also have a new camera called the DAVIS \cite{DAVIS}

\subsection{Asynchronous Time-based Image Sensor (ATIS)}
what is it what does it do \cite{posch2010high}
Example usages Saeed (UWS) \cite{afshar2016investigation}


%%%%%%%%%%%%%%%%%%%%%%%%    EVENT-BASED PROCESSING     %%%%%%%%%%%%%%%%%%%%%%%%%%%%%%%%%

\pagebreak
\section{Vision processing}   % 1 page
Standard videos and frame based approaches

\pagebreak
\section{Event-based processing}     % 2 page
Where does the idea come from, what is it, 
What other processing has already been done with them (things like Motion cones). 
Example usage of DVS is fast motor control by \cite{delbruck2007fast}
Discuss frame accumulation by other groups
Check out more papers in Amy's paper
Example of event-based visial flow calculation by fitting local plans on incoming events\cite{benosman2014event} (what else is this lab doing?)
Neuromorphic systems are becoming more wide spread \cite{delbruck2014research}
Calculating odometry using event based information (application) \cite{censi2014low}
Event-based plane detection \cite{afshar2016investigation}
Event-based IMU use and the problems with it and what different representations can do to help\cite{fida2015pre}
Real time event-based (IMU) processing \cite{fida2015real}


\subsection{Biologically realistic representations}
Rumelhart's back prop might not actually be biologically realistic but effective in learning regardless \cite{Rumelhart1986}. \\
Disucss how this is more like the eye and the advantages of this model of processing \cite{mahowald1992vlsi}
Converts 2D images to more realistic spike trains \cite{afshar2013ripple}
Discusses the biological realism of using spikes \cite{akolkar2015can}
What does this paper say about representations maybe bio isn't always an answer \cite{fida2015pre}

\pagebreak
\subsection{Temporally accumulated representations}  % 1 page
What other work has been done with decayed representations and how does it affect/influence this project
Temporal surfaces \cite{afshar2016investigation}



%%%%%%%%%%%%%%%%%%%%%%%%%%%%%%  NEURAL NETWORKS   %%%%%%%%%%%%%%%%%%%%%%%%%%%%%%%

\pagebreak
\section{Neural Networks}     % 1 page
Use in image processing recently
Made popular by advents in learning agorithms **cite lecun**, learning rules  \cite{Rumelhart1986} and datasets ** cite datasets**

\subsection{Neuron types}
Different neurons have different properties ...
Need for non-linear neurons

\subsubsection{Perceptron}
Brief history, successes and problems

\subsubsection{Sigmoid}
Improvements over perceptrons

\subsubsection{ReLU}
Improvements over Sigmoid

\subsubsection{LSTM (if used)}


\pagebreak
\subsection{Learning rules}
Learning representations by back--propagating errors\cite{Rumelhart1986}
Interesting learning rules for SNNs \cite{Bichler} not exaimined here though.

\subsubsection{Stochastic Gradient Descent (S.G.D.)}

\subsubsection{Adam Optimiser}
Adam: A Method for Stochastic Optimization \cite{kingma2014adam}

\subsubsection{Backpropagation}

\pagebreak

\subsection{Shallow networks}  

\subsubsection{Auto-Encoders}  % 1 page
Who made these?

\subsubsection{Convolutional networks}

\pagebreak
\subsection{Deeper networks}   % 1 page
ResNet etc. % TODO find resnet reference
This was some deep network work \cite{pedroni2013neuromorphic}
Oconor used deep networks \cite{OConnor2013}

\pagebreak
\subsection{Recurrent networks}  % 1 page
Elman and his work\cite{elman1990}
Also some work done by Sutskever and Hinton with recurrent boltzmann machines (might be the motion data? if not should quote that too)\cite{sutskever2009recurrent}

\pagebreak
\subsection{Spiking networks}    % 1 page
O'Connors work on spiking deep belief networks \cite{OConnor2013}
Used a spiking network to train feature detectors \cite{afshar2016investigation}
Move to spiking networks accompanied by converstions from 2D to 1D temporal trains \cite{afshar2013ripple}
Discussion on 2D images and converting to spikes vs just using spiking sensor and the information preserved \cite{akolkar2015can}
Paper on spiking networks for vision tasks, usese the DVS and compares to convNets \cite{martin2015spiking}
Feeding a DVS output directly into spiking NN \cite{Bichler}

\pagebreak
\subsection{Software frameworks}   % 1 page
Caffe\cite{jia2014caffe} Theano Torch7 Tensorflow CNTK

\section{Neural Networks with event-based data}
What is the standard approach, what have people tried and what is lacking. \cite{OConnor2013}  ** TOBI CAR WITH CONVOLUTIONS ** ** UWS temporal surfaces paper if published now ** ** AMYs work ** (who does amy reference?)
Using event-based data and feeding it into a network (Saeed) \cite{afshar2016investigation}

\subsection{Benchmark datasets}
Some benchmark datasets have been given \cite{Gibson2014} and **OTTHER** but represent complex strucutre in the real world.
Creates own plane dataset \cite{afshar2016investigation}
Creates own letter (on rolling board) and DMD projector dataset \cite{akolkar2015can}
Newest dataset of motion and visual navigation, emphasises the need for dataset \cite{barranco2016dataset}

%%%%%%%%%%%%%%%%%%%%%%%%%%%%%%%%%    SUMMARY    %%%%%%%%%%%%%%%%%%%%%%%%%%%%%%%%%%%%%

\section{Literature summary}      % 1 page
Revisit each section in a sentance or two and link them all together.

\subsubsection{What are these references and why did I grab them?}
Evolving probabilistic spiking neural networks for spatio-temporal pattern recognition: A preliminary study on moving object recognition \cite{kasabov2011evolving}

Neuromorphic adaptations of restricted boltzmann machines and deep belief networks, These guys (in particular Pedroni) had something cool, should find out what it was again \cite{pedroni2013neuromorphic}

Probably full of good info for referencing \cite{delbruck2014research}

Double check the use of this paper \cite{gil2014active}

How does this paper fit? applications? what are they doing? temporal surfaces maybe? \cite{davide2014high} another from the same author, seems to have the trend of very fast motor control \cite{mueggler2014event} again high speed object tracking (diff authors) \cite{saner2014high} More somewhat unrelated, must just be about the methods of processing \cite{mueggler2015continuous}, \cite{barranco2016dataset}, This one sounds like cool processing (similar to UWS i think) \cite{Bichler}
