\chapter{Attentional Networks}
\label{ch:attentional}
To help simplify the learning problem the task was again reframed to improve the signal to noise ratio.
Rather than applying decay to the whole image at uniform time intervals and using that as input the to the network each event was decayed around resulting in many more (albiet similar) training examples.
However only an 11x11 area around each event was considered meaning the signal to noise ratio was much higher. 

%%%%%%%%%%%%%%%%%%%%%%%%%%%%%%      ATTENTIONAL NN    %%%%%%%%%%%%%%%%%%%%%%%%%%%%%%%%%%%%%%
\section{Attentional Directly Connected (ADC) Network}



%%%%%%%%%%%%%%%%%%%%%%%%%%%%%%      ATTENTIONAL NN    %%%%%%%%%%%%%%%%%%%%%%%%%%%%%%%%%%%%%%
\section{Attentional Hidden Layer (AHL) Network}

Discuss results of using a hidden layer and the improvements if any


%%%%%%%%%%%%%%%%%%%%%%%%%%%%%%      AUTO ENCODE     %%%%%%%%%%%%%%%%%%%%%%%%%%%%%%%%%%%%%%
\section{Auto Encoder}
How much could be cleaned up
Retry other networks with cleaned up data.

